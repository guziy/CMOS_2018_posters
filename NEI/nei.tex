%%%%%%%%%%%%%%%%%%%%%%%%%%%%%%%%%%%%%%%%%
% a0poster Landscape Poster
% LaTeX Template
% Version 1.0 (22/06/13)
%
% The a0poster class was created by:
% Gerlinde Kettl and Matthias Weiser (tex@kettl.de)
%
% This template has been downloaded from:
% http://www.LaTeXTemplates.com
%
% License:
% CC BY-NC-SA 3.0 (http://creativecommons.org/licenses/by-nc-sa/3.0/)
%
%%%%%%%%%%%%%%%%%%%%%%%%%%%%%%%%%%%%%%%%%

%----------------------------------------------------------------------------------------
%	PACKAGES AND OTHER DOCUMENT CONFIGURATIONS
%----------------------------------------------------------------------------------------

\documentclass[a0b,landscape]{a0poster}

\usepackage{multicol} % This is so we can have multiple columns of text side-by-side
\columnsep=70pt % This is the amount of white space between the columns in the poster
\columnseprule=3pt % This is the thickness of the black line between the columns in the poster

\usepackage[svgnames]{xcolor} % Specify colors by their 'svgnames', for a full list of all colors available see here: http://www.latextemplates.com/svgnames-colors

\usepackage{times} % Use the times font
%\usepackage{palatino} % Uncomment to use the Palatino font

\usepackage{graphicx} % Required for including images
\usepackage{graphbox}
\graphicspath{{figures/}} % Location of the graphics files
\usepackage{booktabs} % Top and bottom rules for table
\usepackage[font=footnotesize,labelfont=bf]{caption} % Required for specifying captions to tables and figures
\usepackage{amsfonts, amsmath, amsthm, amssymb} % For math fonts, symbols and environments
\usepackage{wrapfig} % Allows wrapping text around tables and figures
\usepackage[utf8]{inputenc} % Pour utiliser les caractères accentués
\usepackage[T1]{fontenc}
\usepackage{tikz}

\usepackage{hyperref}
\hypersetup{
  colorlinks=false
}

\usepackage[round]{natbib}
\bibliographystyle{agu}

\usepackage{tabularx}

\usetikzlibrary{shapes,snakes}
\usetikzlibrary{positioning}
\usepackage[export]{adjustbox}
\usepackage[skins,listings,breakable,listingsutf8,theorems,hooks,fitting]{tcolorbox}
\tcbuselibrary{raster}

\newcommand{\hspc}{-0.5cm}
\newcommand{\wspc}{-0.5cm}

\begin{document}

\captionsetup{justification=raggedright}

%----------------------------------------------------------------------------------------
%	POSTER HEADER
%----------------------------------------------------------------------------------------

% The header is divided into three boxes:
% The first is 55% wide and houses the title, subtitle, names and university/organization
% The second is 25% wide and houses contact information
% The third is 19% wide and houses a logo for your university/organization or a photo of you
% The widths of these boxes can be easily edited to accommodate your content as you see fit

\noindent\begin{minipage}[b]{\linewidth}
\centering
\noindent \huge \color{NavyBlue} \textbf{Collective assessment of a set of improvements to surface process parameterizations in CRCM5 over Western Canada} \color{Black}\\[0.15cm] % Title
\noindent\begin{minipage}[c]{0.25\linewidth}
      \center
      \includegraphics[width=7cm, align=c]{logo_cnrcwp.png} \includegraphics[width=7cm, align=c]{nserc_narrow} \includegraphics[width=7cm,align=c]{compute_canada_transparent_small}% Logo or a photo of you, adjust its dimensions here
\end{minipage}
%
\hfill
%
\begin{minipage}[c]{0.45\linewidth}
  \center
  \large \textbf{Huziy O., Sushama L., Teufel B., Duarte L., Ruman C.} \\[0.25cm]
  \large \texttt{Contact email: guziy.sasha@gmail.com}
\end{minipage}
%
\hfill
%
\begin{minipage}[c]{0.25\linewidth}
  \center
  \includegraphics[width=10cm,align=c]{mcgill_logo.png} \includegraphics[width=8cm,align=c]{logo_uqam.png}   % Logo or a photo of you, adjust its dimensions here
\end{minipage}
%
\rule{\linewidth}{3pt}
\end{minipage}
%

%\vspace{0.25cm} % A bit of extra whitespace between the header and poster content

%----------------------------------------------------------------------------------------

\begin{multicols*}{4} % This is how many columns your poster will be broken into, a poster with many figures may benefit from less columns whereas a text-heavy poster benefits from more

%----------------------------------------------------------------------------------------
%	INTRODUCTION
%----------------------------------------------------------------------------------------

%\color{SaddleBrown} % SaddleBrown color for the introduction

\section*{(A) Introduction}
Western Canada, with glaciers and complex topography presents significant
challenges for regional climate models. Many surface-related parameterizations
were improved or newly implemented in the Canadian Regional Climate Model \citep[and references therein]{huziy2017a} to
improve the representation of the surface climate and hydrology. These include
dynamic vegetation, dynamic glaciers, lake-river system and frozen soil
hydraulic conductivity parameterizations. The objective of this study is to
evaluate the impact of these improvements on the simulated climate of western
Canada. The collective evaluation on the model behaviour is assessed through
three current climate (1980–2010) simulations. Two of these simulations are
performed at 0.44$^\circ$ resolution, with and without improvements, while the third
one is performed at 0.11$^\circ$, with all improvements. All the simulations are driven
by ERA-Interim reanalysis at the boundaries.

Simulations with the modified version of the model demonstrate improvements in
the simulated climate, particularly at 0.11$^\circ$ resolution for 2-m air temperature
and at 0.44$^\circ$ for mean spring total precipitation, when the temperature and
precipitation biases get reduced by up to 4 $^\circ$C and 1 mm/day respectively in some
regions. Analysis of temperature and precipitation extremes suggests
improvements, particularly over regions with important orography. Comparison of
simulations at 0.44$^\circ$ and 0.11$^\circ$ resolutions suggest improvements in precipitation
extremes for elevated regions by 0.5–1 mm/day.



%----------------------------------------------------------------------------------------
%	OBJECTIVES
%----------------------------------------------------------------------------------------

% \vspace{0.25cm}
% \begin{tcolorbox}[colback=white,colframe=green!40!black,adjusted title={Main objectives}]
%   \begin{enumerate}
%   \item Evaluate the net impact of the improved CRCM5 model, resulted from adding
%         dynamic vegetation, frozen soil hydraulic conductivity and glacier
%         parameterization, on simulated climate of western Canada at 0.44$^\circ$.
%   \item Identify the impact of increased resolution on the modified model configuration,
%         by comparing simulations at 0.44$^\circ$ and 0.11$^\circ$ horizontal grid
%         spacing.
%   \end{enumerate}
% \end{tcolorbox}

%----------------------------------------------------------------------------------------
%	MATERIALS AND METHODS
%----------------------------------------------------------------------------------------

\section*{(B) Methods and experiment configurations}
%\subsection*{C.1 Methods}
%
This study is based on the comparison of three CRCM5 simulations (Table
\ref{table:simulations}) performed over western Canada. To validate mean and
extreme 2m air temperature and precipitation, daily DAYMET dataset
\citet{thornton1997}, aggregated from the original 1km grid to the corresponding
model resolution (i.e. 0.44$^\circ$ and 0.11$^\circ$) is used.

%
\noindent
\adjustbox{valign=t, minipage=0.5\linewidth}{
  \footnotesize
  \begin{tabularx}{\linewidth}{lX}
  \toprule
  \textbf{Simulation}    & \textbf{Description}\\
  \midrule
  \textbf{WC\_044\_default}    & Default without the enhancements at 0.44$^\circ$\\[0.25cm]
  \textbf{WC\_044\_modified}   & With the enhancements at 0.44$^\circ$ \\[0.25cm]
  \textbf{WC\_011\_modified}   & With the enhancements at 0.11$^\circ$\\
  \bottomrule
  \end{tabularx}
  \captionof{table}{\label{table:simulations}\color{Green} List of simulations used in the current study}
}
%
\hfill
%
\adjustbox{valign=t, minipage=0.46\linewidth}{
  \begin{tcolorbox}[colback=white,colframe=green!40!black, adjusted title={Experiment setup}]
  \flushleft
  \small
  \begin{itemize}
    \item Simulation period: 1980--2010
    \item CRCM5 time step: 5 min (0.11$^\circ$) and 20 min (0.44$^\circ$);
    \item Surface scheme: CLASS;
    \item Lateral boundary conditions: ERA-Interim, 0.75$^\circ$
  \end{itemize}
  \end{tcolorbox}
}

\subsection*{B.1 Enhancements to the default model configuration}

\noindent
\adjustbox{valign=t, minipage=0.48\linewidth}{
\textbf{Dynamic vegetation} enhancement is the module responsible for simulation
of carbon pools and two-way interactions of vegetation and climate, which allows
to simulate inter-annual variability of vegetation properties as well as its
structure, i.e. temporal evolution of extents of various vegetation types.
The model used for this parameterization is the Canadian Terrestrial Ecosystem
Model \citep[CTEM, ][]{melton2016}.
}
%
\hfill
%
\noindent
\adjustbox{valign=t, minipage=0.48\linewidth}{
\textbf{Hydraulic conductivity of frozen soil} is modified based on \citet{ganji2017}. It takes into account frozen water fraction in the
formula for soil hydraulic conductivity and as a result reduces it by a factor of $(1 + \theta_i \cdot C_k)^{-4}$ ($\theta_i$ - volumetric
fraction of frozen water in soil, $C_k$ - constant parameter). This modification is expected to decrease soil hydraulic conductivity during transition seasons (i.e. spring and summer) and
lead to enhancement of surface runoff.
}

\vspace{0.5cm}
\noindent
\adjustbox{valign=t, minipage=0.61\linewidth}{
    \textbf{Dynamic glaciers}. The implementation of dynamic glaciers is inspired by \citet{kotlarski2009}. The parameterization
    uses high-resolution elevation datasets to divide each grid into elevation bins, where glaciers are distributed, and uses
    precipitation and energy fluxes provided by CRCM5 to simulate the evolution of glacier parameters.
    This modification is expected to improve the representation of the spatial extent of glaciers and lead to
    more realistic albedo and therefore a better partitioning of radiation fluxes.
}
%
\hfill
%
\adjustbox{valign=t, minipage=0.35\linewidth}{
  \begin{center}
    \includegraphics[width=0.8\linewidth]{figures/glacier_fraction}
    \captionof{figure}{\label{figure:glacier_fraction}\color{Green} Spatial extent of glacier fractions used as initial conditions for
                        the WC\_044\_modified and WC\_011\_modified configurations. Based on Randolph
                        Glacier Inventory (RGI) global dataset.}

  \end{center}
}






%----------------------------------------------------------------------------------------
%	RESULTS
%----------------------------------------------------------------------------------------

\section*{(C) Results}
\subsection*{C.1 Impact of added parameterizations on seasonal mean fields}
\noindent
\begin{minipage}[t]{\linewidth}

\begin{center}
  \begin{tikzpicture}
      % schematics
      \node (wc044default) {
        \includegraphics[width=0.8\linewidth, trim={0.3cm 0.63cm 0 0.1cm}, clip]{figures/seasonal_mean_TT_bias_wc_044_default}
      };
      \node[left=0.8cm of wc044default, rotate=90, anchor=north, scale=0.8] (wc044default_label) {WC\_044\_default};

      \node [below=-0.5cm of wc044default, anchor=north] (wc044modified) {
        \includegraphics[width=0.8\linewidth, trim={0.3cm 0.1cm 0 0.41cm}, clip]{figures/seasonal_mean_TT_bias_wc_044_modified}
      };
      \node[left=0.8cm of wc044modified, anchor=north, rotate=90, scale=0.8] (wc044modified_label) {WC\_044\_modified};

      \node[below=-1cm of wc044modified.south, anchor=north, text width=0.8\linewidth]{
          \captionof{figure}{\label{figure:t2mvalidation}
                         \color{Green} Seasonal 2m air temperature biases with respect to aggregated
                          DAYMET dataset ($^\circ$C) for WC\_044\_default (first row) and
                          WC\_044\_modified (second row).}
      };

  \end{tikzpicture}

\end{center}
\end{minipage}

\noindent
\begin{minipage}[t]{\linewidth}
\begin{center}
  \begin{tikzpicture}
      \node (wc044default) {
        \includegraphics[width=0.8\linewidth, trim={0.30cm 0.83cm 0 0.1cm}, clip]{figures/seasonal_mean_PR_bias_wc_044_default}
      };
      \node[left=0.8cm of wc044default, rotate=90, anchor=north, scale=0.8] (wc044default_label) {WC\_044\_default};

      \node [below=-1cm of wc044default, anchor=north] (wc044modified) {
        \includegraphics[width=0.8\linewidth, trim={0.30cm 0.1cm 0 0.43cm}, clip]{figures/seasonal_mean_PR_bias_wc_044_modified}
      };
      \node[left=0.8cm of wc044modified, anchor=north, rotate=90, scale=0.8] (wc044modified_label) {WC\_044\_modified};
      \node[below=-1cm of wc044modified.south, anchor=north, text width=0.8\linewidth]{
        \captionof{figure}{\label{figure:prvalidation}\color{Green} Same as
                           Figure~\ref{figure:t2mvalidation} but for total precipitation (mm/day).}
      };
  \end{tikzpicture}

\end{center}
\end{minipage}


%Discussion
\begin{itemize}
  \item DJF and SON 2m-air temperature and total precipitation are not significantly
  affected by the new parameterizations (See Figures \ref{figure:t2mvalidation} and \ref{figure:prvalidation}).

  \item The colder bias in WC\_044\_modified during MAM is caused by the slower growth rate in the vegetation scheme, which leads to higher albedos and cooler temperatures during MAM.

  \item Although MAM season air temperature cold bias is larger in WC\_044\_modified with respect to WC\_044\_default,
  significant decreases of wet biases in MAM total precipitation in the southwestern part of the domain are noted for  WC\_044\_modified.

  \item The new parameterizations produce warmer 2m-air temperatures in the southern part
  of the domain leading to larger positive biases in the southwestern and removing negative biases in
  southeastern parts of the domain.
  \item JJA precipitation is worse in WC\_044\_modified with respect to
  the default configuration in the southeastern part of the domain. This could be explained by the compensation of errors, which
  occurred in WC\_044\_default, where wet bias from spring suppressed the dry biases in summer. But in WC\_044\_modified the
  MMA wet bias is corrected and the soil is not wet enough to eliminate the dry bias during JJA season.
\end{itemize}

\subsection*{C.2 Impact of resolution on seasonal mean 2m-air temperature and total precipitation simulated by the improved model}
\noindent
\begin{minipage}[t]{\linewidth}
In this section the seasonal 2m air temperature and precipitation biases of
WC\_011\_modified are discussed and compared with the biases of
WC\_044\_modified (shown in the previous section,
Figs.~\ref{figure:t2mvalidation}--\ref{figure:prvalidation}).
\begin{center}
\begin{tikzpicture}
      \node (wc011modified) {
        \includegraphics[width=0.8\linewidth, trim={0.3cm 0.1cm 0 0.1cm}, clip]{figures/seasonal_mean_TT_bias_wc_011_modified}
      };
      \node[left=0.8cm of wc011modified, rotate=90, anchor=north, scale=0.8] (wc011modified_label) {WC\_011\_modified};
      \node[below= -1cm of wc011modified.south west,text width=0.8\linewidth, align=left, anchor=north west]{
        \captionof{figure}{\label{figure:t2mvalidation011}\color{Green} Same as the second
                           row in Figure \ref{figure:t2mvalidation} but for WC\_011\_modified ($^\circ$C).}
      };
\end{tikzpicture}
\end{center}
\end{minipage}


\noindent
\begin{minipage}[t]{\linewidth}
\centering
\begin{tikzpicture}
      \node (wc011modified) {
        \includegraphics[width=0.8\linewidth, trim={0.33cm 0.1cm 0 0.1cm}, clip]{figures/seasonal_mean_PR_bias_wc_011_modified}
      };
      \node[left=0.8cm of wc011modified, rotate=90, anchor=north, scale=0.8] (wc011modified_label) {WC\_011\_modified};

      \node[below=-1cm of wc011modified.south west, anchor=north west, text width=0.8\linewidth, align=flush left]{
        \captionof{figure}{\label{figure:prvalidation011}\color{Green} Same as the second
                           row in Figure \ref{figure:prvalidation} but for WC\_011\_modified (mm/day).}
      };
\end{tikzpicture}
\end{minipage}

\begin{itemize}
  \item Increasing resolution of the modified configuration from 0.44$^\circ$ to
        0.11$^\circ$ leads to warmer 2m air temperatures and decreased DJF, MAM and
        SON temperature biases. Although the positive JJA biases in the southern part
        of the domain are amplified in WC\_011\_modified with respect to
        WC\_044\_modified.

  \item The positive MAM precipitation bias in the southern part of the domain,
        which was fixed in WC\_044\_modified reappears in WC\_011\_modified and
        eliminates the dry bias in JJA. The MAM total precipitation overestimation in WC\_011\_modified could
        be caused by warmer temperature, which allow more intensive recycling of soil moisture.

  \item Total precipitation is overestimated in WC\_011\_modified over northern parts of
        the domain for all seasons with respect to DAYMET dataset. Although the
        reliability of the observation dataset in the northern region should be taken
        into consideration given the scarcity of the station data in the northern parts
        of the domain.
\end{itemize}


\subsection*{C.3 Impact of added parameterizations and of resolution on simulated daily extremes}

To evaluate impacts of the model modifications on simulated daily extremes the 90th percentile of
daily minimum temperature (TN90), 10th percentile of daily maximum temperature (TX10) and 90th percentile of
daily mean total precipitation (PR90) are averaged over land and the time series are compared for WC\_044\_default,
WC\_044\_modified and WC\_011\_modified. The indices are calculated as in \citet{curry2016}.

\noindent
\begin{minipage}[t]{\linewidth}
\center
\begin{tikzpicture}
    % schematics
    \node[label={[yshift=-0.5cm]\small TX10}] (tx10) {
      \includegraphics[width=0.48\linewidth, trim={0 0 0 0}, clip]{figures/tx10_ts}
    };
    \node[right=0.1cm of tx10.east, anchor=west, label={[yshift=-0.5cm]\small TN90}] (tn90) {
      \includegraphics[width=0.48\linewidth, trim={0 0 0 0}, clip]{figures/tn90_ts}
    };

    \node[label={[yshift=-0.5cm]\small PR90}, below=0.5cm of tx10.south, anchor=north] (pr90) {
      \includegraphics[width=0.48\linewidth, trim={0 0 0 0}, clip]{figures/pr90_ts}
    };

    \node [below=-1cm of tn90.south, anchor=north, text width=0.4\linewidth, align=left]{
      \captionof{figure}{\color{Green} Area averaged biases of TX10, TN90 and PR90 for WC\_044\_default (blue),
      WC\_044\_modified (orange) and WC\_011\_modified (green). The biases are calculated using DAYMET observation dataset.
      }
    };
\end{tikzpicture}
\end{minipage}

\begin{itemize}
 \item The slower growth rate of the vegetation in spring appears to be corrected
       by the increased resolution in WC\_011\_modified compared to WC\_044\_modified.
 \item WC\_011\_modified, due to better representation of surface characteristics,
       simulates more accurately TX10 and TN90 throughout a year with respect to
       WC\_044\_modified and WC\_044\_default.
 \item As discussed in the previous section, the over recycling of soil moisture in spring causes wet biases in WC\_011\_modified.
\end{itemize}

\subsection*{C.4 Zonal structure of the daily extremes in a region with complex topography}

\noindent
\adjustbox{valign=t, minipage=0.2\linewidth}{
  \includegraphics[width=\linewidth, trim={0 0 2.1cm 0}, clip]{figures/domain_and_focus_for_meridional_avg}
}
%
\hfill
%
\adjustbox{valign=t, minipage=0.25\linewidth}{
  \captionof{figure}{\color{Green}\label{topofig} Topography field (m) is shown in colors.
  Black rectangle is showing the region of meridional averaging.}
}
%
\hfill
%
\adjustbox{valign=t, minipage=0.5\linewidth}{
  To assess performance of the model configurations in the environment with
  complex topography, extreme indices, averaged along meridians, over land points
  inside the region shown in Fig.~\ref{topofig} are analyzed.
}

\vspace{0.15cm}
\noindent
\begin{minipage}[t]{\linewidth}
\begin{center}
\begin{tikzpicture}
    % TX10
    \node[label={[yshift=-0.5cm]\small TX10 ($^\circ$C)}, label={[rotate=90, anchor=south, yshift=-0.5cm]left:\small DJF}] (tx10_djf) {
      \includegraphics[width=0.31\linewidth, trim={0 8cm 0 0cm}, clip]{figures/meridional_avg/DJF_t_air_2m_daily_max}
    };


    \node[label={[rotate=90, anchor=south, yshift=-0.5cm]left:\small MAM}, below=\hspc of tx10_djf] (tx10_mam) {
      \includegraphics[width=0.31\linewidth, trim={0 8cm 0 0cm}, clip]{figures/meridional_avg/MAM_t_air_2m_daily_max}
    };

    \node[label={[rotate=90, anchor=south, yshift=-0.5cm]left:\small JJA}, below=\hspc of tx10_mam] (tx10_jja) {
      \includegraphics[width=0.31\linewidth, trim={0 8cm 0 0cm}, clip]{figures/meridional_avg/JJA_t_air_2m_daily_max}
    };

    \node[label={[rotate=90, anchor=south, xshift=3cm, yshift=-0.5cm]left:\small SON},below=\hspc of tx10_jja] (tx10_son) {
      \includegraphics[width=0.31\linewidth, trim={0 0cm 0 0cm}, clip]{figures/meridional_avg/SON_t_air_2m_daily_max}
    };


    %TN90
    \node[label={[yshift=-0.5cm]\small TN90 ($^\circ$C)}, right=\wspc of tx10_djf.east, anchor=west] (tn90_djf) {
      \includegraphics[width=0.31\linewidth, trim={0 8cm 0 0cm}, clip]{figures/meridional_avg/DJF_t_air_2m_daily_min}
    };

    \node[below=\hspc of tn90_djf] (tn90_mam) {
      \includegraphics[width=0.31\linewidth, trim={0 8cm 0 0cm}, clip]{figures/meridional_avg/MAM_t_air_2m_daily_min}
    };

    \node[below=\hspc of tn90_mam] (tn90_jja) {
      \includegraphics[width=0.31\linewidth, trim={0 8cm 0 0cm}, clip]{figures/meridional_avg/JJA_t_air_2m_daily_min}
    };

    \node[below=\hspc of tn90_jja] (tn90_son) {
      \includegraphics[width=0.31\linewidth, trim={0 0cm 0 0cm}, clip]{figures/meridional_avg/SON_t_air_2m_daily_min}
    };

    %PR90
    \node[label={[yshift=-0.5cm]\small PR90 (mm/day)}, right=\wspc of tn90_djf] (pr90_djf) {
      \includegraphics[width=0.31\linewidth, trim={0 8cm 0 0cm}, clip]{figures/meridional_avg/DJF_total_prec}
    };

    \node[below=\hspc of pr90_djf] (pr90_mam) {
      \includegraphics[width=0.31\linewidth, trim={0 8cm 0 0cm}, clip]{figures/meridional_avg/MAM_total_prec}
    };

    \node[below=\hspc of pr90_mam] (pr90_jja) {
      \includegraphics[width=0.31\linewidth, trim={0 8cm 0 0cm}, clip]{figures/meridional_avg/JJA_total_prec}
    };

    \node[below=\hspc of pr90_jja] (pr90_son) {
      \includegraphics[width=0.31\linewidth, trim={0 0cm 0 0cm}, clip]{figures/meridional_avg/SON_total_prec}
    };


    %elevation
    % \node[below=\hspc of tx10_son] (el1) {
    %   \includegraphics[width=0.31\linewidth, trim={0 0 0 11.5cm}, clip]{figures/meridional_avg/MAM_total_prec}
    % };
    %
    % \node[below=\hspc of tn90_son] (el2) {
    %   \includegraphics[width=0.31\linewidth, trim={0 0 0 11.5cm}, clip]{figures/meridional_avg/JJA_total_prec}
    % };
    %
    % \node[below=\hspc of pr90_son] (el3) {
    %   \includegraphics[width=0.31\linewidth, trim={0 0 0 11.5cm}, clip]{figures/meridional_avg/SON_total_prec}
    % };

    \node [right=0.1cm of tx10_son.south east, anchor=west, text width=0.6\linewidth, yshift=0.5cm]{
      \captionof{figure}{\color{Green} Zonal structure of seasonal mean biases of TX10, TN90 and PR90 for WC\_044\_default (blue),
      WC\_044\_modified (orange) and WC\_011\_modified (green). The biases are calculated using DAYMET observation dataset.
      }
    };
\end{tikzpicture}
\end{center}
\end{minipage}

\vspace{0.25cm}
% discussion of the meridional means  of the extremes
\noindent
\begin{minipage}[t]{\linewidth}
    Overall, the best performance in simulating the extreme indices over the complex
            topography is shown by the WC\_011\_modified model configuration. Moreover for TX10 and TN90, WC\_011\_modified performs better
            for all considered latitudes. The WC\_011\_modified is obviously the worst only for PR90 in the west part of the considered box,
            i.e. upwind of the complex topography region, where the total precipitation is overestimated with respect to the DAYMET dataset.
\end{minipage}



%----------------------------------------------------------------------------------------
%	CONCLUSIONS
%----------------------------------------------------------------------------------------
{
  \color{SaddleBrown} % SaddleBrown color for the conclusions to make them stand out

  \section*{(D) Conclusions}

  \begin{itemize}
  \item In this project, CRCM5 simulations with newly added surface parameterizations of dynamic vegetation,
  glaciers and frozen soil hydraulic conductivity were evaluated over the region of western Canada at 0.44$^\circ$ and 0.11$^\circ$ resolutions.
  \item To fully take advantage of the introduced model enhancements it is advised to use it at 0.11$^\circ$ resolution. In case the
        correct representation of precipitation is required, then the triggering
        mechanism for the convective precipitation generation should be changed to reduce its sensitivity to instabilities and intensity of the generated precipitation.
  \item It is recognized that the modified configurations are not tuned as extensively as the default one and therefore
  some added value is still expected to be gained after the model is retuned.
  \end{itemize}
}
%\color{DarkSlateGray} % Set the color back to DarkSlateGray for the rest of the content

%----------------------------------------------------------------------------------------
%	REFERENCES
%----------------------------------------------------------------------------------------
%\nocite{*} % Print all references regardless of whether they were cited in the poster or not
\footnotesize
%\bibliographystyle{plain} % Plain referencing style
\bibliography{sample} % Use the example bibliography file sample.bib
%
%----------------------------------------------------------------------------------------
%	ACKNOWLEDGEMENTS
%----------------------------------------------------------------------------------------
% \section*{Acknowledgements}
% \footnotesize
% This research was carried out within the Canadian Network for Regional Climate and Weather Processes (CNRCWP) project funded by the Natural Sciences and Engineering Research Council (NSERC) of Canada. Computing resources provided by Compute Canada.

%----------------------------------------------------------------------------------------
\end{multicols*}
\end{document}
