%%%%%%%%%%%%%%%%%%%%%%%%%%%%%%%%%%%%%%%%%
% a0poster Landscape Poster
% LaTeX Template
% Version 1.0 (22/06/13)
%
% The a0poster class was created by:
% Gerlinde Kettl and Matthias Weiser (tex@kettl.de)
%
% This template has been downloaded from:
% http://www.LaTeXTemplates.com
%
% License:
% CC BY-NC-SA 3.0 (http://creativecommons.org/licenses/by-nc-sa/3.0/)
%
%%%%%%%%%%%%%%%%%%%%%%%%%%%%%%%%%%%%%%%%%

%----------------------------------------------------------------------------------------
%	PACKAGES AND OTHER DOCUMENT CONFIGURATIONS
%----------------------------------------------------------------------------------------

\documentclass[a0b,landscape]{a0poster}

\usepackage{multicol} % This is so we can have multiple columns of text side-by-side
\columnsep=70pt % This is the amount of white space between the columns in the poster
\columnseprule=3pt % This is the thickness of the black line between the columns in the poster

\usepackage[svgnames]{xcolor} % Specify colors by their 'svgnames', for a full list of all colors available see here: http://www.latextemplates.com/svgnames-colors

\usepackage{times} % Use the times font
%\usepackage{palatino} % Uncomment to use the Palatino font

\usepackage{graphicx} % Required for including images
\usepackage{graphbox}
\graphicspath{{figures/}} % Location of the graphics files
\usepackage{booktabs} % Top and bottom rules for table
\usepackage[font=footnotesize,labelfont=bf]{caption} % Required for specifying captions to tables and figures
\usepackage{amsfonts, amsmath, amsthm, amssymb} % For math fonts, symbols and environments
\usepackage{wrapfig} % Allows wrapping text around tables and figures
\usepackage[utf8]{inputenc} % Pour utiliser les caractères accentués
\usepackage[T1]{fontenc}
\usepackage{tikz}

\usepackage{hyperref}
\hypersetup{
  colorlinks=false
}
\usepackage[round]{natbib}
\bibliographystyle{agu}

\usepackage{tabularx}

\usetikzlibrary{shapes,snakes}
\usetikzlibrary{positioning}
\usepackage[export]{adjustbox}
\usepackage[skins,listings,breakable,listingsutf8,theorems,hooks,fitting]{tcolorbox}
\tcbuselibrary{raster}
\begin{document}

\captionsetup{justification=raggedright}

%----------------------------------------------------------------------------------------
%	POSTER HEADER
%----------------------------------------------------------------------------------------

% The header is divided into three boxes:
% The first is 55% wide and houses the title, subtitle, names and university/organization
% The second is 25% wide and houses contact information
% The third is 19% wide and houses a logo for your university/organization or a photo of you
% The widths of these boxes can be easily edited to accommodate your content as you see fit

\noindent\begin{minipage}[b]{\linewidth}
\centering
\noindent \huge \color{NavyBlue} \textbf{Heavy lake-effect snowfall events for the Laurentian Great Lakes region for current and future climates} \color{Black}\\[0.25cm] % Title
\noindent\begin{minipage}[c]{0.25\linewidth}
      \center
      \includegraphics[width=8cm, align=c]{logo_cnrcwp.png} \includegraphics[width=8cm, align=c]{nserc_narrow} \includegraphics[width=8cm,align=c]{compute_canada_transparent_small} % Logo or a photo of you, adjust its dimensions here
\end{minipage}
%
\hfill
%
\begin{minipage}[c]{0.45\linewidth}
  \center
  \large \textbf{Huziy O., Sushama L., Leon L., Yerubandi R.} \\[0.5cm]
  \large \texttt{Contact email: guziy.sasha@gmail.com}
\end{minipage}
%
\hfill
%
\begin{minipage}[c]{0.25\linewidth}
  \center
  \includegraphics[width=10cm,align=c]{mcgill_logo.png} \includegraphics[width=8cm,align=c]{logo_uqam.png}  % Logo or a photo of you, adjust its dimensions here
\end{minipage}

%
\rule{\linewidth}{3pt}
\end{minipage}
%

%\vspace{0.25cm} % A bit of extra whitespace between the header and poster content

%----------------------------------------------------------------------------------------

\begin{multicols*}{4} % This is how many columns your poster will be broken into, a poster with many figures may benefit from less columns whereas a text-heavy poster benefits from more

%----------------------------------------------------------------------------------------
%	INTRODUCTION
%----------------------------------------------------------------------------------------

%\color{SaddleBrown} % SaddleBrown color for the introduction

\section*{(A) Introduction}
Lakes are important components of the climate system and can affect regional
climate by modulating surface albedo, surface energy and moisture budgets.
Therefore, they should be realistically represented in climate models. Many
climate models are currently representing lakes interactively using 1D models.
However, for large lakes such as the Laurentian Great Lakes, 3D models are
required, as it is important to simulate the circulation patterns which can
impact lake temperature as well as ice onset melt dates and fractional coverage,
and therefore heavy lake-effect snow (HLES) as suggested by recent studies. The
aim of this study is to compare HLES simulated by a regional climate model
(CRCM5: Canadian Regional Climate Model) with 1D and 3D models for the Great
Lakes, and to assess projected changes to HLES in a future warmer climate.

For comparing the impact of 3D lakes, two CRCM5 simulations, using 1D (Hostetler
model) and 3D (NEMO) lake models, driven by ERA-Interim reanalysis are performed
and analysed over the Great Lakes region for the 1979–2012 period at 10 km
resolution. Lake ice cover is overestimated in the simulation with 1D lake model
leading to reduced HLES due to the reduced source of moisture for HLES events.
Simulated HLES events are greatly improved with the 3D model. Projected changes
to HLES are assessed by comparing CRCM5 simulations for the 2079–2100 future and
1989–2010 current periods, driven by CanESM2. Analysis of projected changes to
heavy lake effect snowfall suggests mostly decreases both in the amounts and
frequencies which could be explained by the increased temperature, leading to
rain rather than snow, and reduced frequency of the cold air outbreaks
triggering these extreme events.


%----------------------------------------------------------------------------------------
%	OBJECTIVES
%----------------------------------------------------------------------------------------

%\color{DarkSlateGray} % DarkSlateGray color for the rest of the content
\vspace{1cm}
\begin{tcolorbox}[colback=white,colframe=green!40!black,adjusted title={Main objectives}]
  \begin{enumerate}
  \item Evaluate the impact of the improved Great Lakes representation on simulated climate and heavy lake-effect snowfall (HLES) events in particular.
  \item Study the climatology of HLES based on gridded observation datasets.
  \item Link HLES events to large scale circulation patterns.
  \end{enumerate}
\end{tcolorbox}

%----------------------------------------------------------------------------------------
%	MATERIALS AND METHODS
%----------------------------------------------------------------------------------------

\section*{(B) Methods and experiment configurations}
%\subsection*{C.1 Methods}
%
This study is based on the comparison of two CRCM5 simulations (Table \ref{table:simulations}) performed over the Laurentian Great Lakes (Fig.~\ref{figure:simdomain}).\\[1cm]
%
\adjustbox{valign=t, minipage=0.45\linewidth}{
    \centering
    \includegraphics[width=\linewidth]{figures/domain_and_experiments}
    \captionof{figure}{\label{figure:simdomain}\color{Green} Simulation domain and configuration schemes}
}
%
\hfill
%
\adjustbox{valign=t, minipage=0.50\linewidth}{
  \footnotesize
  \begin{tabularx}{\linewidth}{lX}
  \toprule
  \textbf{Simulation}    & \textbf{Description}\\
  \midrule
  \textbf{CRCM5\_HL}     & ERA-Interim (0.75$^\circ$) reanalysis-driven, all lakes are modelled by Hostetler (1D) lake model \\[0.5cm]
  \textbf{CRCM5\_NEMO}   & ERA-Interim (0.75$^\circ$) reanalysis-driven, the Great Lakes are modelled by NEMO (3D) ocean model, other lakes in the region are modelled by Hostetler (1D) lake model \\
  \bottomrule
  \end{tabularx}
  \captionof{table}{\label{table:simulations}\color{Green} List of simulations used in the current study}
}

\vspace{0.5cm}
\noindent
\adjustbox{valign=m, minipage=\linewidth}{
  \begin{tcolorbox}[colback=white,colframe=green!40!black, adjusted title={Model configuration}]
  \flushleft
  \small
  \begin{itemize}
    \item Simulation period: 1979--2010
    \item Horizontal grid size and resolution: 452$\times$260, \textbf{0.1$^\circ$};
    \item CRCM5 time step: 5 min;
    \item 1D Lake model: Hostetler;
    \item NEMO model is used in the Great Lakes in CRCM5\_NEMO, where LIM3 is used for lake ice;
    \begin{itemize}
        \footnotesize
        \item Distribution of ice over thickness categories is one of LIM3 prognostic
              variables (in addition to the ice temperature and velocity), which is used to
              calculate ice concentration in a grid cell as a weighted mean over the thickness bins using the
              modelled distribution.
    \end{itemize}


    \item NEMO time step: 30 min;
    \item Surface scheme: CLASS3.5;
    \item Lateral boundary conditions: ERA-Interim, 0.75$^\circ$
  \end{itemize}
  \end{tcolorbox}
}

\vspace{0.5cm}
\noindent
\textbf{HLES: Observations}\\
To diagnose observed HLES, we follow \citet{notaro2015} using the following gridded datasets:
    \begin{itemize}
        \item Temperature and precipitation, required to calculate snowfall, considered
              to be heavy if greater than 10 cm/day, are taken from high resolution datasets
              for Canada  \citep[10km, ][]{hopkinson2011} and the U.S. \citep[0.125$^\circ$, ][]{maurer2002}. Snowfall is examined within the 100km region around the shores of the Great Lakes and it is also
              required to be greater than the mean over the non-local region of the  by 4 cm/day, i.e. locally enhanced.
        \item Ice concentration, required to eliminate impact of frozen lakes, i.e. if ice
              concentration is greater than 70\%, are taken from the extended Great Lakes ice
              cover atlas datasets available from Great Lakes Environmental Research
              Laboratory (GLERL) website.
        \item Wind components, used to do back trajectories, are taken from ERA-Interim reanalysis (0.75$^\circ$).
    \end{itemize}

\noindent
\textbf{HLES: model}\\
The HLES events are diagnosed from the model outputs using the same approach as above, i.e. \citet{notaro2015}, applied to simulated wind components, snow fall and lake ice fraction.

%----------------------------------------------------------------------------------------
%	RESULTS
%----------------------------------------------------------------------------------------

\section*{(C) Results}
\subsection*{C.1 Validation: 2m air temperature, lake surface water temperature and surface currents}

%%%2m air temp and LSWT validation
\begin{center}
    \includegraphics[width=\linewidth]{figures/t2m_bias_crcm5_hl_vs_crcm5_nemo}
    \captionof{figure}{\label{figure:t2mvalidation}\color{Green} Seasonal 2m air temperature ($^\circ$C) from \citet{hopkinson2011} and \citet{maurer2002} for the 1980-2010 period (first row),
    seasonal mean biases in the CRCM5\_HL (second row) and differences between CRCM5\_NEMO and CRCM5\_HL-simulated 2m air temperature (third row).}
\end{center}

\begin{center}
    \includegraphics[width=\linewidth]{figures/sst_validation_crcm5nemo_crcm5hl}
    \captionof{figure}{\label{figure:sstvalidation}\color{Green} Seasonal lake surface water temperature ($^\circ$C) from NOAA OISST (first row),
    seasonal mean biases in the CRCM5\_HL (second row) and seasonal mean biases in the CRCM5\_NEMO (third row) for the 2003-2010.}
\end{center}

\begin{itemize}
    \item CRCM5\_NEMO decreases positive biases seen in CRCM5\_HL around the lakes during spring and summer (Fig.~\ref{figure:t2mvalidation}).
    Negative CRCM5\_HL biases south of the lakes decreased in winter and fall in CRCM5\_NEMO, although positive biases amplified in some regions (Fig.~\ref{figure:t2mvalidation}).
    \item CRCM5\_NEMO-simulated lake surface water temperature is in better agreement with
          the observations due to the added horizontal circulation, which makes the lake
          cooling and warming inertia more realistic (Fig~\ref{figure:sstvalidation}).
    \item 2-m air temperature differences betweenn CRCM5\_NEMO and
          CRCM5\_HL simulations (Fig.~\ref{figure:t2mvalidation}) are in agreement with the
          differences in lake surface water temperature between these simulations
          (Fig~\ref{figure:sstvalidation}).
\end{itemize}


\adjustbox{valign=t, minipage=0.48\linewidth}{
  \includegraphics[width=\linewidth]{seasonal_biases_lake_ice_fraction_February-March_1980-2010}
  \captionof{figure}{\label{figure:ice2dvalidation}\color{Green} Mean February and March ice fraction from the extended Gret Lakes ice atlas, available from GLERL website, (first row), ice fraction biases in
  the CRCM5\_HL simulation (second row) and ice fraction differences between CRCM5\_NEMO and CRCM5\_HL simulations (third row) for the 1980-2010 period.}
}
%
\hfill
%
\adjustbox{valign=t, minipage=0.48\linewidth}{
  \includegraphics[width=\linewidth]{seasonal_biases_total_prec_Winter-Summer_1980-2010}
  \captionof{figure}{\label{figure:precipvalidation}\color{Green} Seasonal total precipitation (mm/day) from \citet{hopkinson2011} and \citet{maurer2002} for the 1980-2010 period (first row),
  seasonal mean biases in the CRCM5\_HL (second row) and differences between CRCM5\_NEMO and CRCM5\_HL-simulated total precipitation (third row).}
}

\begin{itemize}
  \item Ice cover fraction validation suggests mostly improved and lower lake ice concentrations in CRCM5\_NEMO with respect to CRCM5\_HL (Fig.~\ref{figure:ice2dvalidation}).
  The ice cover is mostly underestimated in CRCM5\_NEMO.
  \item Precipitation biases are shown for winter and summer season, when the precipitation differences between CRCM5\_NEMO and CRCM5\_HL are highest (Fig.~\ref{figure:precipvalidation}).
  The positive impact of the 3D lake parameterization in winter is due to higher water temperatures and lower ice cover fractions which lead to increased moistening of the overlying air.
  In summer, the effect on precipitation os opposite due to cooler lake surfaces (Fig~\ref{figure:sstvalidation}).
\end{itemize}




\noindent
\adjustbox{valign=t, minipage=\linewidth}{
  \noindent
  \includegraphics[width=\linewidth]{figures/crcm5_nemo_surface_currents}
  \captionof{figure}{\color{Green} Mean seasonal surface currents (m/s) simulated by CRCM5\_NEMO for the 1980-2010 period.}
}
\begin{itemize}
  \item Strongest surface currents are noted in winter, which corresponds to the result in \citet{dupont2012}.
  \item Keweenav current (southern part of lake Superior in summer) is captured with the maximum speed
  between 6-8 cm/s (maximum speed in the current simulated by \citet{dupont2012} is 7 cm/s).
\end{itemize}


\subsection*{C.2 Impacts of the 3D parametrization of the Great Lakes on simulated HLES: spatial distribution and inter-annual variability}
\noindent
\begin{minipage}{\linewidth}
  \center
  \includegraphics[width=\linewidth]{figures/hles_clim_NDJ}
  \captionof{figure}{\color{Green}\label{figure:hlesspatial} Mean number of HLES days during November-January for the 1980-2010 period derived from observations (left),
  CRCM5\_NEMO (middle) and CRCM5\_HL (right) simulation outputs.}
\end{minipage} \\[0.5cm]


\begin{itemize}
  \item The locations of dominant HLES are captured in CRCM5\_NEMO, although the spatial extent is underestimated east of lakes Superior and Huron.
        This could be due to biases in the model or due to the lower resolution of the wind fields (0.75$^\circ$) used to diagnose observed HLES (Fig.~\ref{figure:hlesspatial}).
  \item The locations and extent of the HLES regions is captured better around the lakes Erie and Ontario, although the values are slightly overestimated in
        CRCM5\_NEMO (Fig.~\ref{figure:hlesspatial}).
\end{itemize}



\adjustbox{valign=t, minipage=0.6\linewidth}{
  \includegraphics[width=\linewidth]{area_avg_boxplots_snow_fall}
  \captionof{figure}{\color{Green} \label{figure:hlesinterannual} Distributions of area-averaged HLES (cm) during November-January
  from observations, CRCM5\_NEMO and CRCM5\_HL respectively for the 1980-2010 period.}
}
%
\hfill
%
\adjustbox{valign=t, minipage=0.4\linewidth}{
\noindent
\begin{itemize}
  \item Simulated mean and inter-annual variability of HLES is improved in CRCM5\_NEMO, although it is still underestimated (Fig.~\ref{figure:hlesinterannual}).
  The correlation of CRCM5\_NEMO (CRCM5\_HL)-based HLES with the observation-based HLES is 0.72 (-0.10).
\end{itemize}
}


\noindent
\adjustbox{valign=t, minipage=0.55\linewidth}{
  \includegraphics[width=\linewidth]{hles_histo_all_m11_12_1}
  \captionof{figure}{\color{Green} \label{figure:hlesmonthly} Monthly area-averaged HLES (\%) during November,
  December and January from observations, CRCM5\_NEMO and CRCM5\_HL respectively for the 1980-2010 period.}
}
%
\hfill
%
\adjustbox{valign=t, minipage=0.45\linewidth}{
  \begin{itemize}
      \item Distributions of HLES over months within the period of interest (Nov-Jan, Fig.~\ref{figure:hlesmonthly})
      suggest that the fractions of HLES from observations and CRCM5\_NEMO are similar and increase from November to January. The distribution of HLES for CRCM5\_HL
      is quite different from the observed one, due to
      the negative lake water temperature biases in winter (Fig.~\ref{figure:sstvalidation}) and overestimated lake ice fractions (Fig.~\ref{figure:ice2dvalidation}), which lead
      to the shift of the majority of the HLES events to the onset period.
  \end{itemize}
}

\subsection*{C.3 Linking HLES to large scale circulation patterns}

\noindent
\adjustbox{valign=t, minipage=0.55\linewidth}{
  \centering
  \includegraphics[width=\linewidth]{hles_wind_compoosits}
  \captionof{figure}{\color{Green} \label{figure:hles_high_minus_low_composite} Differences between the composites of sea level pressure (surface wind) from ERA-Interim reanalysis based
  on mean values for years with
  extreme high and low HLES. The extreme high (low) HLES years selected for the analysis are 1993, 1995, 1998 (1997, 2001, 2006).}
}
%
\hfill
%
\adjustbox{valign=t, minipage=0.45\linewidth}{
\begin{itemize}
  \item Extreme high (low) HLES events are associated with the following two
        surface circulation anomalies: weakening (amplification) of the Aleutian low
        and cyclonic (anticyclonic) anomaly over Hudson Bay. These anomalies
        (anticyclonic over Aleutian islands and cyclonic over northern part of Hudson
        Bay) favor and steer the north westerly flow of cold air in the direction of
        the Great Lakes, thereby favoring the occurrences of HLES
        (Fig.~\ref{figure:hles_high_minus_low_composite}).
\end{itemize}
}

%----------------------------------------------------------------------------------------
%	CONCLUSIONS
%----------------------------------------------------------------------------------------
{
  \color{SaddleBrown} % SaddleBrown color for the conclusions to make them stand out

  \section*{(D) Conclusions}

  \begin{itemize}
  \item In this project, CRCM5 simulations with a 1D and a new 3D parameterizations of the Great Lakes were evaluated with observations and used to analyze HLES events in the region.
  \item The new large lakes parameterization has impact on seasonal 2m air temperature and total precipitation
        around the lakes and it mostly decreases simulation biases with respect to the
        1D parameterization of large lakes.
  \item Surface currents simulated by the coupled system are reasonable and have similar features with previous studies.
  \item Simulated HLES values are greatly improved in CRCM5\_NEMO with respect to
        CRCM5\_HL, due to improved simulated lake surface water temperature and lake
        ice, which highlights the importance of large lake parameterization different
        from that for smaller lakes for better representation of the physical processes
        leading to HLES events.
  \item In future, it is planned to perform transient climate change simulations with
        the improved 3D large lake parameterization to study the projected changes to
        HLES characteristics.
  \end{itemize}
}
%\color{DarkSlateGray} % Set the color back to DarkSlateGray for the rest of the content

%----------------------------------------------------------------------------------------
%	REFERENCES
%----------------------------------------------------------------------------------------
%\nocite{*} % Print all references regardless of whether they were cited in the poster or not
\footnotesize
%\bibliographystyle{plain} % Plain referencing style
\bibliography{sample} % Use the example bibliography file sample.bib
%
%----------------------------------------------------------------------------------------
%	ACKNOWLEDGEMENTS
%----------------------------------------------------------------------------------------
% \section*{Acknowledgements}
% \footnotesize
% This research was carried out within the Canadian Network for Regional Climate and Weather Processes (CNRCWP) project funded by the Natural Sciences and Engineering Research Council (NSERC) of Canada. Computing resources provided by Compute Canada.

%----------------------------------------------------------------------------------------
\end{multicols*}
\end{document}
